\documentclass{letter}
\usepackage{hyperref}
\signature{The Authors}
\address{}
\begin{document}

\begin{letter}{}
\opening{Dear Reviewers:}

We are glad that you both liked the paper and appreciate the helpful comments. We believe we have addressed all of your feedback by making changes to the paper. Below we detail the changes we made.

\section{Reviewer 1}

Your question about friction is important. An ollie can be performed without any foot-board friction at the front foot, but you be able to ollie very high. In the associated MSc thesis appendix Fig. 29(a), we show simulation of an ollie with low foot-board friction that achieves a lower max height. After the ollie was invented, skaters began applying more grippy surfaces to the deck and today's decks are covered with a high grip sandpaper. The front edge of the skater's front shoe rapidly wears down from the friction. Two of the authors of this paper are skateboarders and can personally attest to the role front foot to board friction plays in performing quality ollies. Removing the grip tape trying to do it barefoot are fraught,, for example. To improve the paper, we have added the time histories of the front and rear foot friction magnitude to each of the three simulation figures. By examining \(\dot{s}_1,\dot{s}_2\) and the friction curves you can see when static and dynamic friction are occurring. We've added some commentary in the text describing the friction forces during the different phases of the first simulation.

\section{Reviewer 2}

Point 1

\begin{quote}
    Introduction ``The maneuver can be deconstructed into the six distinct phases shown in Fig. 1'' There are 11 t * s in Fig.1. To show six phases, I prefer to use a table like below. Please consider.
\end{quote}

We have updated the table with a column for the phase number as suggested.

Point 2

\begin{quote}
    Does ``nose-tail symmetry'' mean the symmetry around the center of gravity along the longitudinal direction? Probably, I misunderstood. Fig.3 does not look like the symmetry around the center of gravity along the longitudinal direction.
\end{quote}
We have changed the sentence to say: ``The board was modeled as a simplified popsicle stick skateboard where we assume the board is geometrically symmetric, i.e. the front nose and truck is a mirror of the back tail and truck, and there is no deck concavity.''

Point 3

\begin{quote}
    “During wheel-ground contact, we use a sliding joint for the rear wheel contact to eliminate the ground reaction forces from the equations of motion (EoMs).” Probably, I misunderstood. Were the ground reaction forces in the vertical direction considered?
\end{quote}

We changed the sentence to ``During the wheel-ground contact phase, we use a sliding joint for the rear wheel ground contact to avoid having to expose the ground reaction forces in the \glspl{eom} derivation.'' The ground reaction forces are present, but they are non-contributing forces in that phase.

Point 4

\begin{quote}
    Caption Fig. 3 CoM \verb|->| CoM (Center of Mass) Probably, here is the first appearance of CoM.
\end{quote}

We spell out center of mass on the first appearance in the captions and the first appearance in the text.

Point 5

\begin{quote}
    5) 2.8 Optimal Control Scenarios
\end{quote}

We do not have any quantitative criteria for judging in the resulting ollie simulation is realistic, e.g. comparison to measured data. We did iteratively improve our model until the solutions showed motion that resembles the motion of the ollie depicted in Figure 1 and that the range of the force magnitudes were of the same order of magnitude of those reported in measurements. We changed the first sentence to:

``We solved the base skateboard OCP to demonstrate that the model and optimization methodology produce ollies with qualitatively similar motion to Fig. 1.''

so that the reader is not expecting quantitative criteria.

For the later two points we now reference the rows of Table 1 for clairty:

``To demonstrate simultaneously solving for the control trajectories and optimizing the geometry parameters we present two single parameter optimizations: rows 3 and 4 in Table 1 which have their wheelbase and tail length optimized, respectively, and row 5 which demonstrates a multiple parameter...''

Point 6

\begin{quote}
    6) 3.1 Base Skateboard Optimization ``Immediately prior to the impact of the tail with the ground, the skateboard rapidly moves backward.'' With respect to the human?
\end{quote}

We have added ``with respect to the human''.

\closing{Sincerely,}

\end{letter}
\end{document}