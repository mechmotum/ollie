\section{Conclusion}
\noindent A model is made to optimize the skateboard for ollie height with a human controller. Optimal board shapes have been found that show higher ollie performance than the current Popsicle stick skateboard. The research question 
\begin{quote}
\textit{What are the optimal geometric and inertial parameters of a skateboard for an Olympic athlete to reach maximal ollie height? 
}\end{quote}
has not been answered fully but a closer approximation is given towards the optimal shape for maximal ollie height and more insight is gained in the dynamics of the ollie. Though, the process of creating a model to optimize the skateboard ollie has been an exploration of the endless variables in the movements of both athlete and board. One conclusion is the fact that the created model turned out to be surprisingly close to the real world. The model is a user friendly and quick tool to find optimal board shapes dependent on the kinetics of a human performer. The kinetics can easily be implemented by any researcher or any skateboarder that is in the possession of a force plate. Making this model a very agile and useful tool for skaters, skateboard manufacturers and future researchers. Although the research question is not fully answered, the outcome of all the different optimizations gives a lot of insight in the dynamics of the ollie. This insight could be an inspiration to other researchers, skateboarders and board builders to expand and develop the academic comprehension of the dynamics of skateboarding. 